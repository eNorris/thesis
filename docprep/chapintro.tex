%%
%% This is file `chapmin.tex',
%% generated with the docstrip utility.
%%
%% The original source files were:
%%
%% ths.dtx  (with options: `chapmin')
%% 
%% IMPORTANT NOTICE:
%% 
%% For the copyright see the source file.
%% 
%% Any modified versions of this file must be renamed
%% with new filenames distinct from chapmin.tex.
%% 
%% For distribution of the original source see the terms
%% for copying and modification in the file ths.dtx.
%% 
%% This generated file may be distributed as long as the
%% original source files, as listed above, are part of the
%% same distribution. (The sources need not necessarily be
%% in the same archive or directory.)


Computed tomography (CT) is becoming increasingly pervasive in medical diagnostics because improved algorithms are giving doctors more access to patient information. Increased information results in faster and more accurate diagnosis. However, the CT operation itself gives a radiation dose to the patient which carries some associated risk. A more detailed measure of that risk would help doctors make informed decisions regarding whether a CT scan is warannted, and if so, which type. This can also help doctors estimate spatial dose distribution to the patient to ensure no specific organ received more dose than permissible.

Another application is in treatment.

Currently, no algorithm exists to provide this information.

The goal of this work is to provide such an algorithm.

There are two primary avaialble for the methodologies: Monte Carlo and deterministic methods. Monte Carlo has proven to be very slow in these areas. Furhter, the 3D Cartesian mesh generated from CT reconstruction is well suited to deterministic computation. 

This work forcuses on the discrete ordinates method and presents an implementation specially developed for dose estimation from a medical CT scan. 

\endinput
%%
%% End of file `chapmin.tex'.
