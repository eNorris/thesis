%%
%% This is file `chapmin.tex',
%% generated with the docstrip utility.
%%
%% The original source files were:
%%
%% ths.dtx  (with options: `chapmin')
%% 
%% IMPORTANT NOTICE:
%% 
%% For the copyright see the source file.
%% 
%% Any modified versions of this file must be renamed
%% with new filenames distinct from chapmin.tex.
%% 
%% For distribution of the original source see the terms
%% for copying and modification in the file ths.dtx.
%% 
%% This generated file may be distributed as long as the
%% original source files, as listed above, are part of the
%% same distribution. (The sources need not necessarily be
%% in the same archive or directory.)


Computed tomography (CT) is becoming increasingly pervasive in medical diagnostics because improved algorithms are giving doctors more access to patient information. Increased information results in faster and more accurate diagnosis. However, the CT operation itself gives a radiation dose to the patient which carries some associated risk. A more detailed measure of that risk would help doctors make informed decisions regarding whether a CT scan is warannted, and if so, which type. This can also help doctors estimate spatial dose distribution to the patient to ensure no specific organ received more dose than permissible.

Currently, dose estimation relies on \textit{a priori} computation verified with a standardized benchmark. No methodology currently exists to verify that the dosimetry evaluation was accurate after the patient has undergone the procedure. This work proposes a methodology by which a patient's CT reconstruction is used to compute the dose received from the radiation beam.

Another, similar, application is dose estimation of patients receiving radiation therapy. Often, before the procedure, a time dependent, 4D CT scan of the patient is taken so that radiologists can account for breathing patterns during administration of the treatment. 

ESTIMATE. PHANTOM. CORRELATE.

BASED ON REFERENCE PHANTOM.

ADD FIGURE

Once the procedure (whether diagnostic or treatment) is completed, the dose distribution is assumed to follow the empirical model.

There are two primary methodologies available to provide the transport solution: Monte Carlo and deterministic methods. Monte Carlo has proven to be very slow in these areas. Further, the 3D Cartesian mesh generated from CT reconstruction is well suited to deterministic computation. 

This work forcuses on the discrete ordinates method and presents an implementation specially developed for dose estimation from a medical CT scan. 

Computational mesh from a sinogram. Reconstruction.
The DOCTORS code takes a computational reconstruction mesh of CT numbers, cross section data, and other parameters related to the solution methodology as input. The output is the collided and uncollided flux as well as the dose in each computational cell. These values can be compared to the empirical correlation results as a verification that the patient received an appropriate dose.

SHOULD BE FAST - GPU

USER INTERFACE

\endinput
%%
%% End of file `chapmin.tex'.
