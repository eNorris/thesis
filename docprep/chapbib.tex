%%
%% This is file `chapbib.tex',
%% generated with the docstrip utility.
%%
%% The original source files were:
%%
%% ths.dtx  (with options: `chapmin,addbib')
%% 
%% IMPORTANT NOTICE:
%% 
%% For the copyright see the source file.
%% 
%% Any modified versions of this file must be renamed
%% with new filenames distinct from chapbib.tex.
%% 
%% For distribution of the original source see the terms
%% for copying and modification in the file ths.dtx.
%% 
%% This generated file may be distributed as long as the
%% original source files, as listed above, are part of the
%% same distribution. (The sources need not necessarily be
%% in the same archive or directory.)

This chapter summarizes the literature reviewed in preparation of this work. The literature review is split into multiple sections, each highlighting relevant works pertaining to a particular component or section of this work. The first two sections cover introductory work and steps necessary for preprocessing data. Section~\ref{sec:discordlit} covers historical and recent work in discrete ordinate solution methodology, the methodology used by DOCTORS. The remaining sections cover GPU hardware and other implementation facets.

\section{CT Dose Estimation}
The dose a patient receives form both diagnostic CT procedures and radiation therapy is of great concern to the medical community [CITE]. 

MOTIVATION

Monte Carlo codes have been used in the past to compute flux-to-dose conversion factors for a reference person [CITE ICRP].

COMPUTATION

\section{CT Phantom Generation}
The CT phantom is populated with CT numbers, also known as Hounsfield numbers. These values are related to the attenuation coefficient of the material represented by the voxel. Traditionally, a CT number of zero corresponds to the attenuation of water while -1000 corresponds to dry air CITE though some variations do exist CITE.

Many authors have provided correlations that map the CT number to both a material and a density.

First ones.
Saw \citep{ref:sawc} did stuff. And \citep{ref:plessisf}. Then \citet{ref:schneideru} said stuff. And \citet{ref:kimh} did stuff with GEANT4.

Another Schneider did some stuff too \citep{ref:schneiderw}. And Vanderstraeten has a long name \citep{ref:vanderstraetenb}.

The new guy provides 19 materials. The density is measured.
The DOCTORS code relies on the CT nuber to material mapping of \citet{ref:ottossonr} who developed a 19-group BLAH. Table~\ref{table:ctmap} shows the mapping of CT number to material composition (reproduced from \citep{ref:ottossonr}).

\begin{table}[ht]
\caption{Water Regions}
\centering 
\begin{tabular}{l c r r r r r r r r r r r r}
\hline \hline   
Media       & CT Range      & H    & C    & N    & 0    & Na  & Mg & P    & S   & Cl  & Ar  & K   & Ca \\ [0.5ex] 
\hline
Air         & -950 to -100  &      &      & 75.7 & 23.3 &     &    &      &     &     & 1.3 &     &       \\
Lung        & -1000 to -950 & 10.3 & 10.5 & 3.1  & 74.9 & 0.2 &    & 0.2  & 0.3 & 0.3 &     & 0.2 &       \\
Adipose     & -100 to 15    & 11.2 & 50.8 & 1.2  & 36.4 & 0.1 &    &      & 0.1 & 0.1 &     &     &       \\
Connective  & 15 to 129     & 10.0 & 16.3 & 4.3  & 68.4 & 0.4 &    &      & 0.4 & 0.3 &     &     &       \\
Bone        & 129 to 200    & 9.7  & 44.7 & 2.5  & 35.9 &     &    & 2.3  & 0.2 & 0.1 &     & 1.0 & 4.5   \\  [1ex]
\hline
\end{tabular}
\label{table:ctmap}
\end{table}

Two phantom datasets are used in this work [CITE].

\section{Discrete Ordinates}\label{sec:discordlit}
The dsicrete ordinates method dates back to CITE but remains one of the most prominent solution modalities for ration transport in use today.

A comprehensive review of discrete ordinate methods is given by \citet{ref:lewise}. Additional references include 


The Exnihilo thing by \citet{ref:evanst}. And DeHart wrote in \citep{ref:dehartm}

The Ibrahim did stuff \citep{ref:ibrahima} and Lee did stuff too \citep{ref:leeb}.

With regards to angles, stuff has been done \citep{ref:ahrensc}

Attila had a 1st col src added by \citep{ref:wareingt}. Some rays are better than others \citep{ref:mathewsk}.

Dort and Tort \citep{ref:rhoadesw}.

Denovo.

\section{Raytracing}

Raytracing is important too. Woo came up with on \citep{ref:wooa} which was improved by \citet{ref:liuy} and \citet{ref:hel}.

\section{GPU Acceleration}
GPUs are cool.

CUDA is cool.


\endinput
%%
%% End of file `chapbib.tex'.
