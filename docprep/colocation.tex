\documentclass{article}
\usepackage{graphicx}
\usepackage{bm}
\usepackage{amsmath}
\usepackage{amssymb}
\usepackage[toc,page]{appendix}
\usepackage{listings}
\usepackage{color}
\usepackage{listings}
\usepackage{fullpage}
\usepackage{mathtools}
\numberwithin{equation}{subsection}

\begin{document}

%%---------------------------------------------------------------------------%%
%% collocation_methods.tex
%%---------------------------------------------------------------------------%%

%%---------------------------------------------------------------------------%%
\chapter{Collocation Angular Methods}
\label{cha:coll-angul-meth}
%%---------------------------------------------------------------------------%%

Many deterministic transport methods treat the angular terms on the left-hand
side of Eq.~(\ref{eq:mg-transport}) using a finite-element collocation method.
Foremost among these are discrete ordinates (\Sn), method of characteristics
(MOC), and short characteristics schemes. In each of these methods, the
right-hand side scattering kernel is expanded using Spherical Harmonics.  A
finite element collocation discretization is then applied to the angular
terms. The expansion of the scattering and source terms are described in
\S\S~\ref{sec:angul-discr-scatt-1}--\ref{sec:angul-discr-extern}.  The angular
collocation scheme is explained in \S~\ref{sec:angul-coll-discr}.

%%---------------------------------------------------------------------------%%
\section{Angular Discretization of Scattering Terms}
\label{sec:angul-discr-scatt-1}
%%---------------------------------------------------------------------------%%

Equation~(\ref{eq:mg-transport}) is still an integro-differential equation in
which the principal unknown, $\psi$, is present in the scattering integral.
Additionally, the multigroup scattering cross section is a function of angle.
Accordingly, we need some method for dealing with the angular dependence of
the scattering source.  We tackle this problem by recognizing that
$\sigma_{\text{s}}$ is only a function of the cosine between the incoming and
outgoing angles (the polar scattering angle assuming azimuthal symmetry).
Therefore, we can expand the scattering cross section in Legendre polynomials,
\begin{equation}
  \sigg{}(\vOmega'\cdot\vOmega)=\sum_{l=0}^{N}
  \frac{2l+1}{4\pi}P_{l}(\vOmega'\cdot\vOmega)
  \sigg{l}\:.
  \label{eq:legendre-expansion-scattering}
\end{equation}
Integrating over all angles we calculate the total scattering cross section as
follows:
\begin{equation}
  \begin{split}
    \sigma_{\text{s}}^{gg'} &=
    \int_{4\pi}\sigg{}(\vOmega'\cdot\vOmega)\:d\vOmega \equiv
    2\pi\int_{-1}^{1}\sigg{}(\mu_o)\:d\mu_o\\
    %%
    &=\sum_{l=0}^{N}\frac{2l+1}{2}\sigg{l}\int_{-1}^{1}
    P_0(\mu_o)P_l(\mu_o)\:d\mu_o\\
    %%
    &=\sigg{0}\:,
  \end{split}
\end{equation}
where we have used the orthogonality of Legendre polynomials listed in
Chap.~\ref{cha:math-prop-ident}.

Applying Eq.~(\ref{eq:legendre-expansion-scattering}) in
Eq.~(\ref{eq:mg-transport}) gives a scattering source defined by
\begin{equation}
  q^g_{s}(\vOmega) = \sum_{g'=0}^G \int_{4\pi}
  \sum_{l=0}^N\frac{2l+1}{4\pi}P_{l}(\vOmega'\cdot\vOmega)\sigg{l}
  \psi^{g'}(\vOmega')\:d\vOmega'\:,
  \label{eq:scattering-source-before-addition}
\end{equation}
where we have suppressed the spatial dependence, $\ve{r}$.  The
addition theorem of Spherical Harmonics can be used to evaluate the
Legendre function, $P_l(\vOmega'\cdot\vOmega)$,
\begin{equation}
  P_l(\vOmega'\cdot\vOmega) = \frac{4\pi}{2l+1}\sum_{m=-l}^l
  Y_{lm}(\vOmega)Y^{\ast}_{lm}(\vOmega')\:,
\end{equation}
where the $Y_{lm}$ are defined in Eq.~(\ref{eq:complete-spherical-harmonics}).
The scattering must be real; therefore, we can follow a methodology similar to
the techniques described in Chap.~\ref{cha:spher-harm-expans} that shows how
to expand a real-valued function using complex Spherical Harmonics.  First,
the expansion is split into positive and negative components of $m$,
\begin{equation}
  P_l(\vOmega'\cdot\vOmega) = \frac{4\pi}{2l+1}
  \Bigl[
  Y_{l0}(\vOmega)Y_{l0}(\vOmega') +
  \sum_{m=1}^l
  \bigl(Y_{lm}(\vOmega)Y^{\ast}_{lm}(\vOmega') +
  Y_{l-m}(\vOmega)Y^{\ast}_{l-m}(\vOmega')\bigr)\Bigr]\:.
\end{equation}
Examining the $m=0$ term gives the following result
\begin{equation}
  Y_{l0} = \sqrt{\frac{2l+1}{4\pi}}\,P_{l0} = Y^e_{l0}\:,
\end{equation}
where the $Y^e$ are defined in Eq.~(\ref{eq:Ye}).

Expanding the Spherical Harmonics into real and imaginary components as shown
in Eq.~(\ref{eq:complex-Y}), the sum over $m>0$ becomes
\begin{equation}
  \sum_{m=1}^l\Bigl(
  \hat{Y}^e_{lm}(\vOmega)\hat{Y}^e_{lm}(\vOmega') +
  \hat{Y}^o_{lm}(\vOmega)\hat{Y}^o_{lm}(\vOmega') +
  \hat{Y}^e_{l-m}(\vOmega)\hat{Y}^e_{l-m}(\vOmega') +
  \hat{Y}^o_{l-m}(\vOmega)\hat{Y}^o_{l-m}(\vOmega')\Bigr)\:,
\end{equation}
where the imaginary terms have been set to zero because the scattering must be
real.  Using Eqs.~(\ref{eq:Yem-Ye-m}) and (\ref{eq:Yom-Yo-m}), the summation
becomes
\begin{equation}
  \sum_{m=1}^l\Bigl(
  2\hat{Y}^e_{lm}(\vOmega)\hat{Y}^e_{lm}(\vOmega') +
  2\hat{Y}^o_{lm}(\vOmega)\hat{Y}^o_{lm}(\vOmega')\Bigr)
\end{equation}
Comparing Eqs.~(\ref{eq:hat-Ye}) and (\ref{eq:hat-Yo}) with Eqs.~(\ref{eq:Ye})
and (\ref{eq:Yo}) leads to the following relationships,
\begin{equation}
  \hat{Y}^e_{lm} = \frac{1}{\sqrt{2}}Y^e_{lm}\:,\quad
  \hat{Y}^o_{lm} = \frac{1}{\sqrt{2}}Y^o_{lm}\:.
\end{equation}
After applying these equations in the $m>0$ terms and combining with the $m=0$
term described above, the expression for $P_l(\vOmega\cdot\vOmega')$ is
\begin{equation}
  P_l(\vOmega'\cdot\vOmega) = \frac{4\pi}{2l+1}
  \Bigl[
  Y^e_{l0}(\vOmega)Y^e_{l0}(\vOmega') +
  \sum_{m=1}^l
  \bigl(Y^e_{lm}(\vOmega)Y^e_{lm}(\vOmega') +
  Y^o_{lm}(\vOmega)Y^o_{lm}(\vOmega')\bigr)\Bigr]\:,
  \label{eq:P_l(mu_o)}
\end{equation}
where, as shown in Chap.~\ref{cha:spher-harm-expans}, the $Y^e$ and $Y^o$ form
an orthonormal basis.

Returning to the scattering source defined in
Eq.~(\ref{eq:scattering-source-before-addition}), Eq.~(\ref{eq:P_l(mu_o)})
provides the Legendre polynomial for the cosine of the scattering angle, and
\begin{equation}
    q^g_{s}(\vOmega) = \sum_{g'=0}^G \int_{4\pi}
    \sum_{l=0}^N
    \Bigl[Y^e_{l0}(\vOmega)Y^e_{l0}(\vOmega')
    +\sum_{m=1}^l
    \bigl(Y^e_{lm}(\vOmega)Y^e_{lm}(\vOmega') +
    Y^o_{lm}(\vOmega)Y^o_{lm}(\vOmega')\bigr)\Bigr]
    \sigg{l}
    \psi^{g'}(\vOmega')\:d\vOmega'\:.
\end{equation}
Rearranging terms and defining
\begin{alignat}{3}
  \even_{lm} &= \int_{4\pi}Y^e_{lm}(\vOmega)\psi^g(\vOmega)\:d\vOmega\:,
  \quad& m\ge 0\:,\label{eq:even-flux}\\
  %%
  \odd_{lm} &= \int_{4\pi}Y^o_{lm}(\vOmega)\psi^g(\vOmega)\:d\vOmega\:,
  \quad& m>0\:,\label{eq:odd-flux}
\end{alignat}
which follows directly from Eqs.~(\ref{eq:even_moment}) and
(\ref{eq:odd_moment}), the scattering source becomes
\begin{equation}
  q^g_{s}(\ve{r},\vOmega) = \sum_{g'=0}^G
  \sum_{l=0}^N
  \sigg{l}(\ve{r})
  \Bigl[
  Y^e_{l0}(\vOmega)\evenp_{l0}(\ve{r}) +
  \sum_{m=1}^l
  \bigl(
  Y^e_{lm}(\vOmega)\evenp_{lm}(\ve{r}) +
  Y^o_{lm}(\vOmega)\oddp_{lm}(\ve{r})
  \bigr)\Bigr]\:.
  \label{eq:mg-scattering-source}
\end{equation}
Equation~(\ref{eq:mg-scattering-source}) is the multigroup anisotropic
scattering source that is defined by the order of the Legendre expansion,
$P_N$, of the scattering.  For a given $P_N$ order, $(N+1)^2$ moments are
required to integrate the scattering operator.  The moments in
Eqs.~(\ref{eq:even-flux}) and (\ref{eq:odd-flux}) are the \textit{angular flux
  moments} or, simply, flux moments.

The scalar flux is defined in Eq.(\ref{eq:scalar_flux}) as the zeroth moment
of the angular flux.  Therefore, we have
\begin{equation}
  \phi^g = \int_{4\pi}\psi^g\:d\vOmega = \sqrt{4\pi}\int_{4\pi}Y^e_{00}
  \psi^g\:d\vOmega = \sqrt{4\pi}\even_{00}\:.
\end{equation}
The current is defined
\begin{equation}
  \begin{split}
    \ve{J}^g &= \int_{4\pi}\Bigl[\mu\psi^g\hat{\ve{e}}_x +
    \eta\psi^g\hat{\ve{e}}_y +
    \xi\psi^g\hat{\ve{e}}_z\Bigr]\:d\vOmega\\
    %%
    &=-\sqrt{\frac{4\pi}{3}}\even_{11}\hat{\ve{e}}_x -
    \sqrt{\frac{4\pi}{3}}\odd_{11}\hat{\ve{e}}_y +
    \sqrt{\frac{4\pi}{3}}\even_{10}\hat{\ve{e}}_z\:.
  \end{split}
\end{equation}
More concisely,
\begin{equation}
  J_x^g = -\sqrt{\frac{4\pi}{3}}\even_{11}\:,\quad
  J_y^g = -\sqrt{\frac{4\pi}{3}}\odd_{11}\:,\quad
  J_z^g = \sqrt{\frac{4\pi}{3}}\even_{10}\:,
\end{equation}
for $\ve{J}= J_x\hat{\ve{e}}_x + J_y\hat{\ve{e}}_y
+J_z\hat{\ve{e}}_z$.

%%---------------------------------------------------------------------------%%
\subsection{Scattering Expansions in 2D Geometry}
\label{sec:scatt-expans-2d}
%%---------------------------------------------------------------------------%%

To expand the scattering kernel in 2D geometry, we must define a symmetry
plane.  We choose the $XY$-plane as the symmetry plane; although the
$XZ$-plane could be chosen as well.  The resulting 2D $XY$ coordinate system
is shown in Fig.~\ref{fig:2D_coord}.
\begin{figure}
  \begin{center}
    \input{2D_coord.pdftex_t}
  \end{center}
  \caption{2D coordinate system used in \denovo.  Symmetry is defined about
    the $XY$ plane.}
  \label{fig:2D_coord}
\end{figure}
The anisotropic scattering source is defined in
Eq.~(\ref{eq:mg-scattering-source}).  Using symmetry, the number of terms
required to define this source can be reduced.  The symmetry illustrated in
Fig.~\ref{fig:2D_coord} yields the following constraint equation
\begin{equation}
  \psi(\theta, \varphi) = \psi(\pi - \theta, \varphi)\:,\quad
  0\le\theta\le\frac{\pi}{2}\:.
\end{equation}
With this constraint in mind, we expand the integral in
Eq.~(\ref{eq:even-flux}) as follows
\begin{equation}
  \phi_{lm} = \int_{0}^{2\pi}d\varphi\int_{0}^{\pi}
  Y^e_{lm}(\theta,\varphi)\psi(\theta,\varphi)\sin\theta\,d\theta\:.
  \label{eq:expanded-even-moment}
\end{equation}
The inner integral over the polar angle can be split into 2 components,
\begin{equation}
    \int_{0}^{\pi}
    Y^e_{lm}(\theta,\varphi)\psi(\theta,\varphi)\sin\theta\,d\theta =
    \int_{0}^{\pi/2}Y^e_{lm}(\theta,\varphi)\psi(\theta,\varphi)
    \sin\theta\,d\theta +
    \int_{\pi/2}^{\pi}Y^e_{lm}(\theta,\varphi)\psi(\theta,\varphi)
    \sin\theta\,d\theta\:.
\end{equation}
Transforming variables to $\theta'=\pi-\theta$ in the second integral on the
right-hand side gives
\begin{equation}
  \int_{\pi/2}^{\pi}Y^e_{lm}(\theta,\varphi)\psi(\theta,\varphi)
  \sin\theta\,d\theta = \int_{0}^{\pi/2}
  Y^e_{lm}(\pi-\theta,\varphi)\psi(\pi-\theta,\varphi)
  \sin(\pi-\theta)\,d\theta\:.
\end{equation}
Applying $\sin(\pi-\theta) = \sin\theta$ and the symmetry constraint to the
preceding two equations and substituting into
Eq.~(\ref{eq:expanded-even-moment}), the even angular flux moments become
\begin{equation}
  \phi_{lm} = \int_{0}^{2\pi}d\varphi
  \int_{0}^{\pi/2}\bigl(Y^e_{lm}(\theta,\varphi) +
  Y^e_{lm}(\pi-\theta,\varphi)\bigr)\psi(\theta,\varphi)
    \sin\theta\,d\theta\:.
\end{equation}
Substituting Eq.~(\ref{eq:Ye}) into the preceding expression gives
\begin{equation}
  \phi_{lm} = \int_{0}^{2\pi}d\varphi
  \int_{0}^{\pi/2}
  D_{lm}\cos m\varphi\Bigl(
  P_{lm}(\cos\theta) +
  P_{lm}\bigl(\cos(\pi-\theta)\bigr)
  \Bigr)
  \psi(\theta,\varphi)\sin\theta\,d\theta\:.
\end{equation}
Using $\cos(\pi-\theta)=-\cos\theta$ and applying the following identity
\cite{arfken},
\begin{equation}
  P_{lm}(-x) = (-1)^{l+m}P_{lm}(x)\:,
\end{equation}
we derive the following formula for calculating the even angular flux moments,
\begin{equation}
  \phi_{lm} = \int_{0}^{2\pi}d\varphi
  \int_{0}^{\pi/2}
  \bigl(1 + (-1)^{l+m}\bigr)
  Y^e_{lm}(\theta,\varphi)\psi(\theta,\varphi)\sin\theta\,d\theta\:.
\end{equation}
Using the same procedure for the odd moments, the angular flux moments are
defined
\begin{align}
  \phi_{lm} &= \int_{0}^{2\pi}d\varphi
  \int_{0}^{1}
  \bigl(1 + (-1)^{l+m}\bigr)
  Y^e_{lm}(\theta,\varphi)\psi(\theta,\varphi)\,d\xi\:,
  \label{eq:2D-even-moments}\\
  \vartheta_{lm} &= \int_{0}^{2\pi}d\varphi
  \int_{0}^{1}
  \bigl(1 + (-1)^{l+m}\bigr)
  Y^o_{lm}(\theta,\varphi)\psi(\theta,\varphi)\,d\xi\:,
  \label{eq:2D-odd-moments}
\end{align}
where we have written the integrals over $d\theta$ using the substitution
$\xi=\cos\theta$.  Equations~(\ref{eq:2D-even-moments}) and
(\ref{eq:2D-odd-moments}) are the 2D equivalents of Eqs.~(\ref{eq:even-flux})
and (\ref{eq:odd-flux}) when symmetry is defined about the $XY$ plane.

A brief examination of Eqs.~(\ref{eq:2D-even-moments}) and
(\ref{eq:2D-odd-moments}) reveals that for all odd sums of $l+m$ (ie.
$l+m\in\{2k+1; \forall k\in Z\}$) the angular moments vanish. This results in
$(N+1)(N+2)/2$ angular moments for $P_N$ scattering whereas 3D calculations
require $(N+1)^2$ moments.  Furthermore, we only need to consider quadrature
angles in the $0\le\xi\le 1$ directions and the factor of 2 that results when
$l+m$ is even preserves the integration over $4\pi$; this is a natural result
of symmetry in the polar direction.  Three-dimensional space contains 8
octants that consist of 4 octants for $+\xi$ and 4 octants for $-\xi$.  In 2D
space, only the 4 octants in $+\xi$ are required.  In 2D geometry these 4
octants are referred to as quadrants.

%%---------------------------------------------------------------------------%%
\section{Angular Discretization of External Sources}
\label{sec:angul-discr-extern}
%%---------------------------------------------------------------------------%%

In most cases, the external source will be a known function of angle.  For
example, an isotropic external source is
\begin{equation}
  q^g_e(\vOmega) = \frac{s^g}{4\pi}\:.
\end{equation}
However, in certain cases (e.g. coupling to a $k$-eigenvalue calculation) the
external source is defined in moments.
Equation~(\ref{eq:Spherical_Harmonic_Expansion_f}) gives the Spherical
Harmonics expansion of a real-valued function.  Applying the same methodology
gives the expansion of the external source
\begin{equation}
  q^g_e(\vOmega) = \sum_{l=0}^{N}\Bigl[
  Y^e_{l0}(\vOmega)\qe_{l0} +
  \sum_{m=1}^{l}
  \bigr(
  Y^e_{lm}(\vOmega)\qe_{lm} + Y^o_{lm}(\vOmega)\qo_{lm}\bigr)
  \Bigr]\:,
  \label{eq:mg-external-source}
\end{equation}
where the spatial dependence has been suppressed.  The even and odd source
moments are defined
\begin{alignat}{3}
  \qe_{lm} &= \int_{4\pi}Y^e_{lm}(\vOmega)q^g_e(\vOmega)\:d\vOmega\:,
  \quad&m\ge 0\:,\label{eq:even-source}\\
  \qo_{lm} &= \int_{4\pi}Y^o_{lm}(\vOmega)q^g_e(\vOmega)\:d\vOmega\:,
  \quad&m>0\:.\label{eq:odd-source}
\end{alignat}
Now, using Eqs.(\ref{eq:mg-scattering-source}) and
(\ref{eq:mg-external-source}), we write the entire source as
\begin{equation}
  \begin{split}
    Q^g(\ve{r},\vOmega) &= q^g_s(\ve{r},\vOmega) +
    q^g_e(\ve{r},\vOmega)\\
    %%
    &=\sum_{g'=0}^G
    \sum_{l=0}^N
    \sigg{l}(\ve{r})
    \Bigl[
    Y^e_{l0}(\vOmega)\evenp_{l0}(\ve{r}) +
    \sum_{m=1}^l
    \bigl(
    Y^e_{lm}(\vOmega)\evenp_{lm}(\ve{r}) +
    Y^o_{lm}(\vOmega)\oddp_{lm}(\ve{r})
    \bigr)\Bigr]\\
    %%
    &\quad+
    \sum_{l=0}^{N}\Bigl[
    Y^e_{l0}(\vOmega)\qe_{l0}(\ve{r}) +
    \sum_{m=1}^{l}
    \bigl(
    Y^e_{lm}(\vOmega)\qe_{lm}(\ve{r}) +
    Y^o_{lm}(\vOmega)\qo_{lm}(\ve{r})\bigr)
    \Bigr]\:.
  \end{split}
  \label{eq:mg-source}
\end{equation}

%%---------------------------------------------------------------------------%%
\section{Angular Collocation Discretization}
\label{sec:angul-coll-discr}
%%---------------------------------------------------------------------------%%

Combining Eqs.~(\ref{eq:mg-transport}) and (\ref{eq:mg-source}) gives the
multigroup transport equation with the scattering expanded in Spherical
Harmonics,
\begin{equation}
  \hOmega\cdot\grad\psi^g(\vOmega) +
  \sigma^g\psi^g(\vOmega) = Q^g(\vOmega)\:
  \label{eq:mg-transport-scattering+source}
\end{equation}
where the spatial dependence has been suppressed.  While Spherical Harmonics
have been used to expand the scattering sources (and possibly the external
source), we still have a dependence on $\vOmega$ that needs to be resolved.
We apply the discrete ordinates ($S_N$) approximation, which is a collocation
method in angle.  Solving Eq.~(\ref{eq:mg-transport-scattering+source}) at
discrete angular locations requires the following equation,
\begin{equation}
    \hOmega_a\cdot\grad\psi^g_a +
    \sigma^g\psi^g_a
    =\sum_{g'=0}^G
    \sum_{l=0}^N
    \sigg{l}
    \Bigl[
    Y^e_{l0}(\vOmega_a)\evenp_{l0} +
    \sum_{m=1}^l
    \bigl(
    Y^e_{lm}(\vOmega_a)\evenp_{lm} +
    Y^o_{lm}(\vOmega_a)\oddp_{lm}
    \bigr)\Bigr]\\
    + q_e^g(\vOmega_a)\:,
    \label{eq:MG-SN-Equation}
 \end{equation}
where $\psi^g_a \equiv\psi^g(\vOmega_a)$.  The angles are integrated by a
quadrature rule such that
\begin{equation}
  \int_{4\pi}d\vOmega = \sum_{a=1}^{n}w_a = 4\pi\:,
\end{equation}
where $w_a$ are the quadrature weights, and $n$ is the total number of angles.
Different quadrature sets have different numbers of unknowns.  The
Level-Symmetric quadrature set has $n = N(N+2)$ unknowns for an $S_N$
approximation.

Given that many angles result from even low-order $S_N$ approximations, we see
why the source has been expanded in Spherical Harmonics.  Consider, an $S_8$
calculation has 80 angles per unknown location per group.  For a $P_3$
expansion 16 moments are required to define the source, a factor of 5
reduction in memory storage.  An $S_{16}$ calculation could be used for more
accuracy with a $P_3$ calculation resulting in a factor of 18 savings in
memory.

Using the quadrature integration rule the flux moments in
Eqs.~(\ref{eq:even-flux}) and (\ref{eq:odd-flux}) are evaluated using
\begin{align}
  \even_{lm} &= \sum_{a=1}^{n}Y^e_{lm}(\vOmega_a)\psi^g_a w_a\:,
  \label{eq:even-flux-quad-int}\\
  \odd_{lm} &= \sum_{a=1}^{n}Y^o_{lm}(\vOmega_a)\psi^g_a w_a\:.
  \label{eq:odd-flux-quad-int}
\end{align}
Similarly, the source moments in Eqs.~(\ref{eq:even-source}) and
(\ref{eq:odd-source}) are calculated using
\begin{align}
  \qe_{lm} &= \sum_{a=1}^{n}Y^e_{lm}(\vOmega_a)q^g_e(\vOmega_a) w_a\:,\\
  \qo_{lm} &= \sum_{a=1}^{n}Y^o_{lm}(\vOmega_a)q^g_e(\vOmega_a) w_a\:.
\end{align}
The $S_N$ method will be conservative if the quadrature set effectively
integrates the even and odd Spherical Harmonics.  If the orthogonality
conditions in Eqs.~(\ref{eq:Ye-orthogonality}) and (\ref{eq:Yo-orthogonality})
are preserved, then integrating the anisotropic scattering should yield,
\begin{equation}
  \int_{4\pi}q_s^g(\vOmega)\,d\vOmega =
  \sum_{a=1}^{n}q_s^g(\vOmega_a)w_a = \sqrt{4\pi}\sigg{0}\evenp_{00}
    = \sigma^{gg'}_{s}\phi^{g'}\:,
\end{equation}
which will yield a conservative particle balance equation.

%%---------------------------------------------------------------------------%%
\section{Operator Form of the Discrete Ordinates Equation}
\label{sec:oper-form-discr-1}
%%---------------------------------------------------------------------------%%

The multigroup discrete ordinates, or \Sn\, equation is (see
Eq.~(\ref{eq:MG-SN-Equation}))
\begin{multline}
    \hOmega_a\cdot\grad\psi^g_a(\ve{r}) +
    \sigma^g(\ve{r})\psi^g_a(\ve{r})
    \\=\sum_{g'=0}^G
    \sum_{l=0}^N
    \sigg{l}(\ve{r})
    \Bigl[
    Y^e_{l0}(\vOmega_a)\evenp_{l0}(\ve{r})
    +
    \sum_{m=1}^l
    \bigl(
    Y^e_{lm}(\vOmega_a)\evenp_{lm}(\ve{r}) +
    Y^o_{lm}(\vOmega_a)\oddp_{lm}(\ve{r})
    \bigr)\Bigr]
    +
    q_e^g(\ve{r}, \vOmega_a)
    \:.
    \tag{\ref{eq:MG-SN-Equation}}
\end{multline}
We defer the spatial treatment of Eq.~(\ref{eq:MG-SN-Equation}) until
Chap.~\ref{cha:spat-discr-sn}.  This equation can be written using a concise
operator notation that helps illuminate numerical solution techniques.  The
operator form of Eq.~(\ref{eq:MG-SN-Equation}) is
\begin{equation}
  \ve{L}\Psi = \ve{M}\ve{S}\Phi + Q\:.
  \label{eq:operator-transport}
\end{equation}
Here we use the convention that bold letters represent discrete operators or
matrices and script symbols and letters represent vectors.

The sizes of the operators in Eq.~(\ref{eq:operator-transport}) are determined
from the following dimensions:
\begin{equation}
  \begin{aligned}
    N_g &= \text{number of groups}\:,\\
    t   &= \text{number of moments}\:,\\
    n   &= \text{number of angles}\:,\\
    N   &= \text{$P_N$ order}\:,\\
    N_c &= \text{number of cells}\:,\\
    N_e &= \text{number of unknowns per cell}\:.
  \end{aligned}
\end{equation}
Now, we define
\begin{align}
  a &= N_g\times n\times N_c\times N_e\:,\\
  f &= N_g\times t\times N_c\times N_e\:.
\end{align}
Equation~(\ref{eq:operator-transport}) can then be defined in terms of
the sizes of the operators,
\begin{equation}
  (a\times a)(a\times 1) = (a\times f)(f\times f)(f\times 1) +
  (a\times 1)\:.
  \label{eq:operator-sizes}
\end{equation}
More specifically, with the groups defined over the range $g\in[0,G]$,
at each spatial unknown we can write
\begin{equation}
    \ve{L}
    \begin{pmatrix}
      \Psi_0 \\
      \Psi_1 \\
      \Psi_2 \\
      \vdots   \\
      \Psi_G
    \end{pmatrix} =
    \begin{pmatrix}
      \ve{M} & 0 & 0 & 0 & 0 \\
      0 & \ve{M} & 0 & 0 & 0 \\
      0 & 0 & \ve{M} & 0 & 0 \\
      0 & 0 & 0 & \ddots & 0 \\
      0 & 0 & 0 & 0 & \ve{M} \\
    \end{pmatrix}
    %%
    \begin{pmatrix}
      \ve{S}_{00} & \ve{S}_{01} & \ve{S}_{02} & \cdots &
      \ve{S}_{0G} \\
      \ve{S}_{10} & \ve{S}_{11} & \ve{S}_{12} & \cdots &
      \ve{S}_{1G} \\
      \ve{S}_{20} & \ve{S}_{21} & \ve{S}_{22} & \cdots &
      \ve{S}_{2G} \\
      \vdots & \vdots & \vdots & \ddots & \vdots \\
      \ve{S}_{G0} & \ve{S}_{G1} & \ve{S}_{G2} & \cdots &
      \ve{S}_{GG}
    \end{pmatrix}
    \begin{pmatrix}
      \Phi_0 \\
      \Phi_1 \\
      \Phi_2 \\
      \vdots   \\
      \Phi_G
    \end{pmatrix}
    %%
    +
    \begin{pmatrix}
      Q_0 \\
      Q_1 \\
      Q_2 \\
      \vdots   \\
      Q_G
    \end{pmatrix}\:,
    \label{eq:matrix-transport}
\end{equation}
Here, $\Psi_g$ and $Q_g$ are vectors of size $n$,
\begin{align}
  \Psi_g &= \begin{pmatrix}
    \psi^g_1 & \psi^g_2 & \psi^g_3 & \cdots \psi^g_n
  \end{pmatrix}^T\:,\\
  Q_g &= \begin{pmatrix}
    Q^g_1 & Q^g_2 & Q^g_3 & \cdots Q^g_n
  \end{pmatrix}^T\:.
\end{align}
The spatial unknowns are implicit in the above matrices and will be
investigated in Chaps.~\ref{cha:spat-discr-sn} and
\ref{cha:transp-solut-meth}.

The operator $\ve{M}$ is the moment-to-discrete matrix. It is used to project
harmonic moments onto discrete angle space, and it is defined
\begin{equation}
  {\small
    \ve{M} = \begin{pmatrix}
      \Ye{00}{1} & \Ye{10}{1} & \Yo{11}{1} & \Ye{11}{1} &
      \Ye{20}{1} & \cdots & \Yo{NN}{1} & \Ye{NN}{1} \\
      \Ye{00}{2} & \Ye{10}{2} & \Yo{11}{2} & \Ye{11}{2} &
      \Ye{20}{2} & \cdots & \Yo{NN}{2} & \Ye{NN}{2} \\
      \Ye{00}{3} & \Ye{10}{3} & \Yo{11}{3} & \Ye{11}{3} &
      \Ye{20}{3} & \cdots & \Yo{NN}{3} & \Ye{NN}{3} \\
      \vdots     & \vdots     & \vdots     & \vdots     &
      \vdots     &        & \vdots     & \vdots     \\
      \Ye{00}{n} & \Ye{10}{n} & \Yo{11}{n} & \Ye{11}{n} &
      \Ye{20}{n} & \cdots & \Yo{NN}{n} & \Ye{NN}{n}
  \end{pmatrix}}\:.
\end{equation}
The moments of the angular flux are calculated from discrete angular fluxes
using the moment-to-discrete matrix,
\begin{equation}
  \phi = \ve{D}\psi\:.
\end{equation}
Clearly, the discrete form of $\ve{D}$ is defined by the quadrature
integration rules in Eqs.~(\ref{eq:even-flux-quad-int}) and
(\ref{eq:odd-flux-quad-int}),
\begin{equation}
    \ve{D} = \ve{M}^{T}\ve{W}\:.
\end{equation}
where $\ve{W}$ is an $(a\times a)$ diagonal matrix of the quadrature weights.
Using the definition of $\ve{M}$ from Eq.~(\ref{eq:operator-sizes}), the size
of $\ve{D}$ can be determined from
\begin{equation}
  \ve{D} \equiv (f\times a)(a\times a) = (f\times a)\:.
\end{equation}
The $\ve{D}$ matrix has the following form
\begin{equation}
  \ve{D} = \begin{pmatrix}
    w_1\Ye{00}{1} & w_2\Ye{00}{2} & w_3\Ye{00}{3} & \cdots & w_n\Ye{00}{n} \\
    w_1\Ye{10}{1} & w_2\Ye{10}{2} & w_3\Ye{10}{3} & \cdots & w_n\Ye{10}{n} \\
    w_1\Yo{11}{1} & w_2\Yo{11}{2} & w_3\Yo{11}{3} & \cdots & w_n\Yo{11}{n} \\
    w_1\Ye{11}{1} & w_2\Ye{11}{2} & w_3\Ye{11}{3} & \cdots & w_n\Ye{11}{n} \\
    w_1\Ye{20}{1} & w_2\Ye{20}{2} & w_3\Ye{20}{3} & \cdots & w_n\Ye{20}{n} \\
    \vdots & \vdots & \vdots & & \vdots \\
    w_1\Yo{NN}{1} & w_2\Yo{NN}{2} & w_3\Yo{NN}{3} & \cdots & w_n\Yo{NN}{n} \\
    w_1\Ye{NN}{1} & w_2\Ye{NN}{2} & w_3\Ye{NN}{3} & \cdots & w_n\Ye{NN}{n}
  \end{pmatrix}\:.
\end{equation}
Also, even though $\ve{M}$ projects angular flux moments onto discrete angular
flux space, in general $\psi\ne\ve{M}\phi$ unless $\ve{M} = \ve{D}^{-1}$.
This condition is met when using the \textit{Galerkin} quadrature set, but
most quadrature sets do not satisfy this requirement. The moments of the
angular flux vector are defined
\begin{equation}
  \Phi_g = \begin{pmatrix}
    \even_{00} & \even_{10} & \odd_{11} & \even_{11} & \even_{20}
    & \cdots & \odd_{NN} & \even_{NN}
  \end{pmatrix}^T\:,
\end{equation}

The scattering cross sections are defined
\begin{equation}
  \ve{S}_{gg'} = \begin{pmatrix}
    \sigg{0} & 0 & 0 & 0 & 0 & 0 & 0 & 0 \\
    0 & \sigg{1} & 0 & 0 & 0 & 0 & 0 & 0 \\
    0 & 0 & \sigg{1} & 0 & 0 & 0 & 0 & 0 \\
    0 & 0 & 0 & \sigg{1} & 0 & 0 & 0 & 0 \\
    0 & 0 & 0 & 0 & \sigg{2} & 0 & 0 & 0 \\
    0 & 0 & 0 & 0 & 0 & \ddots   & 0 & 0 \\
    0 & 0 & 0 & 0 & 0 & 0 & \sigg{N} & 0 \\
    0 & 0 & 0 & 0 & 0 & 0 & 0 & \sigg{N}
  \end{pmatrix}\:.
\end{equation}
Here the block-matrix $\ve{S}_{gg'}$ defines scattering cross sections for
particles that scatter from group $g'$ into group $g$.  The lower triangular
part of $\ve{S}$ represents down-scattering, the diagonal represents in-group
scattering, and the upper diagonal is up-scattering.

%%---------------------------------------------------------------------------%%
%% end of collocation_methods.tex
%%---------------------------------------------------------------------------%%

%%% Local Variables:
%%% mode: latex
%%% TeX-master: "exnihilo"
%%% End:

\end{document}