%%
%% This is file `chapmin.tex',
%% generated with the docstrip utility.
%%
%% The original source files were:
%%
%% ths.dtx  (with options: `chapmin')
%% 
%% IMPORTANT NOTICE:
%% 
%% For the copyright see the source file.
%% 
%% Any modified versions of this file must be renamed
%% with new filenames distinct from chapmin.tex.
%% 
%% For distribution of the original source see the terms
%% for copying and modification in the file ths.dtx.
%% 
%% This generated file may be distributed as long as the
%% original source files, as listed above, are part of the
%% same distribution. (The sources need not necessarily be
%% in the same archive or directory.)


 %% ... sample chapter ...

Conclusions.

\section{First conclusion}

I concluded many things.

\section{Future Work}

A number of simple modifications to DOCTORS could be made that would greatly increase the code's usability and robustness. One of these modifications, generation of more refined group structures, is not a change to DOCTORS \textit{per se} but rather a change in the input cross section data. The other changes are additions to DOCTORS that would add new capabilities to the software.

In addition to specific modifications, some additional, broader future goals can also be identified.

\subsection{Group Structure}

The cross section data currently used by DOCTORS are taken from SCALE6.2. While these cross sections have been found to be sufficient to produce flux distributions in medical CT imaging, a more refined group structure designed for medical applications would be worth investigation.

This work can be done using either NJOY [CITE] or the newly released AMPX [CITE] code. Either code has the capability to collapse an ENDF formatted data file into a group structure of the user's choosing in the format DOCTORS reads.



NJOY or AMPX.

\subsection{Temporal Extension}

Use 4D CT to do temporal stuff as well as spatial stuff.

\subsection{Multileaf Collimation}
Collimators!

\endinput
%%
%% End of file `chapmin.tex'.
