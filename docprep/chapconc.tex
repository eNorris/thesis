%%
%% Edward T. Norris
%% Discrete Ordinates Computed Tomography Organ Dose Simulator (DOCTORS)
%% 
%% === Conclusion ===
%%

This chapter summarizes the conclusions drawn from the results given in Chapter 5 and suggests further improvements to DOCTORS.

\section{First conclusion}

I concluded many things.

\section{Applicability to Clinical Settings}
With further development, DOCTORS is well positioned to become a code of significant clincal impact.

Great effort was put into producing a code that is simple to use yet capable of producing powerful results quickly. However, usage in a clinical setting would likely require further refinement of the GUI to make the code more robust and intuitive. Additionally, the direct output of DOCTORS may not necessarily be of clinical importance since 

\section{Future Work}

A number of simple modifications to DOCTORS could be made that would greatly increase the code's usability and robustness. One of these modifications, generation of more refined group structures, is not a change to DOCTORS \textit{per se} but rather a change in the input cross section data. The other changes are additions to DOCTORS that would add new capabilities to the software.

In addition to specific modifications, some additional, broader future goals can also be identified.

\subsection{Group Structure}

The cross section data currently used by DOCTORS are taken from SCALE6.2. While these cross sections have been found to be sufficient to produce flux distributions in medical CT imaging, a more refined group structure designed for medical applications would be worth investigation.

This work can be done using either NJOY [CITE] or the newly released AMPX [CITE] code. Either code has the capability to collapse an ENDF formatted data file into a group structure of the user's choosing in the format DOCTORS reads.

NJOY or AMPX.

\subsection{Temporal Extension}
Use 4D CT to do temporal stuff as well as spatial stuff.

\subsection{Partial Acceptance Criteria}
Allow it in the raytracer.

\subsection{Multileaf Collimation}
Collimators!

\subsection{Extension to Therapy Beams}
High energy will be more anisotropic. Will need spherical harmonics.

\subsection{Organ Identification}
Currently, DOCTORS cannot automatically identify specific organs, thus it is only able to compute the equivalent dose via dose deposition.

\endinput
%%
%% End of file `chapmin.tex'.
