%%
%% This is file `chapmin.tex',
%% generated with the docstrip utility.
%%
%% The original source files were:
%%
%% ths.dtx  (with options: `chapmin')
%% 
%% IMPORTANT NOTICE:
%% 
%% For the copyright see the source file.
%% 
%% Any modified versions of this file must be renamed
%% with new filenames distinct from chapmin.tex.
%% 
%% For distribution of the original source see the terms
%% for copying and modification in the file ths.dtx.
%% 
%% This generated file may be distributed as long as the
%% original source files, as listed above, are part of the
%% same distribution. (The sources need not necessarily be
%% in the same archive or directory.)


\section{Preprocessing}
Two computational phantoms were used for verification of the DOCTORS code.

Some preprocessing was done to each phantom before it was used in DOCTORS to provide results.

The results from DOCTORS were compared to analagous results from MCNP6. The MCNP6 reference compared to is automatically generated by DOCTORS to guarantee that the geometry and source parameters are identical. This process is described in detail in Section~\ref{sec:mcnpgen}.

\subsection{Artifact Removal}
At the periphery of the computational mesh, artifacts can appear as a byproduct of the reconstruction algorithm used. Figure~\ref{fig:phantomWaterUnmodified} shows the artifacts in question in a $sy$ slice at the lowest ($z = 0$) level in the phantom. Two hundred sixty one (261) artifacts were detected in the phantom. To remove these artifacts, the CT number of any voxels whose CT number was greater than or equal to 65500 was set to 0. Figure~\ref{fig:phantomWaterModified} shows the same cross sectional plot after the artifacts are removed.

The water phantom in Fig.~\ref{fig:phantomWaterModified} can be broken into four regions: the water phantom, the container, the air, and the corner artifacts. The water phantom is the centermost region of the slice

\begin{figure}[tb]
  \begin{center}
   \includegraphics[width=3.75in]{figs/phantomWaterUnmodified}
  \end{center}
  \caption{The unmodified phantom. An $xy$ slice at $z$ index = 0.}
\label{fig:phantomWaterUnmodified}
\end{figure}

\begin{figure}[tb]
  \begin{center}
   \includegraphics[width=3.75in]{figs/phantomWaterModified}
  \end{center}
  \caption{The phantom after artifacts are removed. An $xy$ slice at $z$ index = 0.}
\label{fig:phantomWaterModified}
\end{figure}

\begin{table}[ht]
\caption{Water Regions}
\centering 
\begin{tabular}{l c r r}
\hline \hline   
Region    & CT Range & Voxels & Fraction (\%)\\ [0.5ex] 
\hline
Water     & $0 \leq x < 67$ & 1581199 & 37.70 \\
Container & $67 \leq x < 600$ & 81502 & 1.94 \\
Air       & $600 \leq x < 1080$ & 1650372  & 39.35 \\
Artifact  & $1080 \leq x < 65535$ & 881231 & 21.01 \\  [1ex]
\hline
\end{tabular}
\label{table:nonlin}
\end{table}

Once the artifacts are removed, the histogram is plotted. Figure~\ref{fig:waterHist} shows two versions of the water histogram, \ref{fig:waterHistLin} on a linear scale and \ref{fig:waterHistLog} on a logarithmic scale.

\begin{figure}
    \centering
    \begin{subfigure}[b]{0.45\textwidth}
        \includegraphics[width=\textwidth]{figs/waterHistogramLin2}
        \caption{Linear}
        \label{fig:waterHistLin}
    \end{subfigure}
    ~
    \begin{subfigure}[b]{0.45\textwidth}
        \includegraphics[width=\textwidth]{figs/waterHistogramLog}
        \caption{Logarithmic}
        \label{fig:waterHistLog}
    \end{subfigure}
    \caption{The histogram.}\label{fig:waterHist}
\end{figure}

\subsection{Gantry Handling}
The gantry is there.

\subsection{Geometry Simplification}
The original phantom data for both the water phantom and the liver phantom are $256\times256\times64$ meshes which result in 4,194,304 total voxels. Though MCNP6 is capable of running such a mesh, the overhead of loading the mesh into memory and initiating the Monte Carlo solver can take on the order of hours. Therefore, the benchmarks were run using a simplified geometry of $64\times64\times16$ which has only 65,536 voxels. The smaller files can run to completion of $10^9$ particles within two hours.

\section{Benchmark}

MCNP is good.

\section{Comparison}

They agree.


\endinput
%%
%% End of file `chapmin.tex'.
