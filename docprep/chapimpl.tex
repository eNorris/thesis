%%
%% Edward T. Norris
%% Discrete Ordinates Computed Tomography Organ Dose Simulator (DOCTORS)
%% 
%% === Implementation ===
%%

This chapter covers the details of the implementation of the different components of the DOCTORS code base. The code is available from a GitHub Git repository. At the time of this writing, the repository is located at \texttt{https://github.com/eNorris/thesis} and is only available subject to special request and license agreement. An executable version is available.

The code is implemented in C++ and requires a C0x11 compliant compiler. The user interface is written using the Qt5 framework. GPU computation is performed via CUDA which must be compiled seperatly by Nvidia's proprietary compiler, \texttt{nvcc}.

INSERT NOTE ABOUT CODE LISTING FORMATTING

\section{Vector Flattening}\label{sec:flatten}

Data is stored in large, multi-dimensional arrays. For example, the CT voxel phantom can be intuitively stored as a 3D array of floating precision values which can be easily indexed. However, DOCTORS is written such that it is extensible to GPUs. GPUs are not optimized for data stored in a multi-dimensional fashion, but rather for data stored as a single 1-dimensional array. To emulate the 3-dimensional storage arrangement of the data, indexing arithmetic is used.

On the CPU, data is stored as C++ \texttt{std::vector<T>} objects where the \texttt{T} template parameter can be either \texttt{float} or \texttt{double} depending on the precision (32 bits or 64 bits respectively) needed. The storage space available to a \texttt{std::vector<T>} object is effectively unlimited, bounded only by the host CPU's physical RAM. However, other effects such as memory fragmentation can limit the practical size of a single, continuous \texttt{std::vector<T>} object significantly. To emulate indexing of a $N_x \times N_y$ matrix, a single \texttt{std::vector<T>} is created and resized to $N_x \times N_y$ elements. The global index, $i$, in the flattened array is then
\begin{equation}
i = i_x N_y + i_y.
\end{equation}
This pattern extends to multi-dimensional matrices. For example, a $N_x \times N_y$ matrix would be globally indexed as
\begin{equation}
i = i_x N_y N_z + i_y N_z + i_z.
\end{equation}

Figure~\ref{fig:indx_ex} gives an example of the spatial indexing scheme using an $8 \times 8 \times 8$ mesh. The indicated voxel has $i_x$, $i_y$, and $i_z$ indices of 1, 5, and 6 respectively. Therefore, the index value of the highlighted voxel is $1 \times 8 \times 8 + 5 \times 8 + 6$ = 110. This flattening pattern continues to higher dimensional phase space to encompas energy and direction.

\begin{figure}[tb]
  \begin{center}
   \includegraphics[width=3.75in]{figs/indx_ex}
  \end{center}
  \caption{Conversion from 3D indexing to 1D "flattened" indexing.}
\label{fig:indx_ex}
\end{figure}

A variable that is dependent on $x$, $y$, $z$, $E$, and $\Omega$ would be written as $\phi(x, y, z, E, \Omega)$ would be discretized to $\phi(i_x, i_y, i_z, G, i_a)$. Rather than storing $\phi$ as a 5-dimensional matrix, $\phi$ is stored as a 1-dimensional vector of the same number of elements.

\section{Cross Section Parsing}\label{sec:xsparse}
AMPX cross sections are stored in a binary format. Figure~\ref{fig:ampx} shows the data format of the overall file. The binary file is split into a sequence of records. The header section of the file has overall information about the file including the number of nuclides stored, their energy group structure, and a comment string describing file. Following the header section is a list of directory records. Each directory has information about a specific nuclide including the cross section reaction types available, temperatures for which the cross sections were evaluated, the number of Legendre expansion coefficients stored in the scatter matrix, and other data. Following the directories, more records follow that describe the energy structure used in the file. There is one such record for each particle type. The libraries used in the current version of DOCTORS have both neutron and photon cross section data stored.

The file header section, directories, and energy group boundaries describe the structural information necessary to read the following nuclide records. Nuclides are listed in the same order their directories are found after the header setion. Each nuclide contains numerous records.

\begin{figure}
    \centering
    \begin{subfigure}[b]{0.45\textwidth}
        \includegraphics[width=\textwidth]{figs/ampx1}
        \caption{}
        \label{fig:ampx1}
    \end{subfigure}
    ~
    \begin{subfigure}[b]{0.45\textwidth}
        \includegraphics[width=\textwidth]{figs/ampx2}
        \caption{}
        \label{fig:ampx2}
    \end{subfigure}
    \caption{The format of the AMPX file. (a) The file contains a header, a list of directory records, energy group information and a list of nuclide entries. (b) Each nuclide entry contains multiple records; the two of concern in this work are the last two containing gamma data.}\label{fig:ampx}
\end{figure}

Each record contains a string of data bytes between a four byte header and four byte footer. The header and footer are identical and, when interpreted as a signed four byte integer, give the size in bytes of the record. An example record is shown in Fig.~\ref{fig:ampxbytes1}.

The data is stored in Big Endian format which must be converted to Little Endian for most Intel and AMD processors.  The endianess rearrangement is shown in Fig.~\ref{fig:ampxbytes2}. Each record is one of 12 unique types; each type of record is formatted and interpreted differently. For example, Type 1 contains general information about the data file such as the number of nuclides, number of energy groups, and a brief description. Type 2 records contains a list of floating-point numbers representing the energy group boundaries. Figure~\ref{fig:ampxbytes1} shows a Type 1 record. In the case of the header bytes given in Fig.~\ref{fig:ampxbytes1} and~\ref{fig:ampxbytes2}, the bytes correspond to the integer 440 which is the length in bytes of the record (not including the header or footer). Each byte of data is sequentially read, reordered, and converted to an apprpriately typed variable. Table~\ref{tab:reorder} shows the converstion for the record given in Fig.~\ref{fig:ampxbytes1}.

\begin{figure}[tb]
  \begin{center}
   \includegraphics[width=4.75in]{figs/ampxbytes1}
  \end{center}
  \caption{An example record parsed. The header and footer are identical and equal to the number of bytes in the record.}
\label{fig:ampxbytes1}
\end{figure}

\begin{figure}[tb]
  \begin{center}
   \includegraphics[width=4.75in]{figs/ampxbytes2}
  \end{center}
  \caption{The byte reordering from Big Endian to Little Endian. Individual bits within a byte are not rearranged, but the ordering of the four bytes withing a 32-bit word is reversed.}
\label{fig:ampxbytes2}
\end{figure}

\begin{table}[ht]
\caption{Byte Reordering}
\centering 
\begin{tabular}{l | c | c | c | l}
  \hline \hline   
  Word  & Big Endian & Little Endian & Interpreted & Notes\\ [0.5ex] % inserts table 
  \hline
  0   & B8 01 00 00 & 00 00 01 B8 & 440    & Header                    \\
  1   & 8B 69 00 00 & 00 00 69 8B & 27019  & ID                        \\
  2   & A4 01 00 00 & 00 00 01 A4 & 420    & Number of nuclides        \\
  ... &      ---    &      ---    &    --- & ---                       \\
  11  & 70 75 6F 43 & 43 6F 75 70 & "Coup" & First four \texttt{char} of the title   \\
  12  & 20 64 65 6C & 6C 65 64 20 & "led " & Second four \texttt{char} of the title  \\
  ... &     ---     &    ---      &  ---   & ---                       \\
  110 & 20 20 20 20 & 20 20 20 20 & "    " & 4 spaces ending the title \\
  111 & B8 01 00 00 & 00 00 01 B8 & 440    & Footer                    \\ 
  [1ex]      % [1ex] adds vertical space
  \hline
\end{tabular}
\label{tab:reorder}
\end{table}

The header always reports the number of 8-bit bytes required to store the data. However, some data types, such as texttt{char}, are only one byte so each word represents multiple (4 in the case of \texttt{char}) distinct characters.


The first entry in each nuclide entry is the directory record. The directory contains general information regarding the nuclide and information necessary to parse the proceeding records. Following the directory, Bondarenko data, resonance parameters, neutron data, and gamma production which are not of interest in photon only problems are read.

The penultimate record contains the average cross section data. This data is averaged over all energies and directions using Eq.~\ref{eq:groupxs}. The only data used in this section in DOCTORS is the total cross section values necessary for implementing the fully discretized form of the LBE given in Eq.~\ref{eq:boltz_i}.

The 2D data is stored in a special format optimized for scatter matrix data. The format concists of a sequence of "magic numbers" interpersed within a list of data points. A magic number is a nine digit number IIIJJJKKK. The first three digits, III, are the group number of the highest energy group to scatter to the energy of interest. The next three digits, JJJ, are the group number of the lowest energy to scatter. The final three digits, KKK, are the group number of the sink energy to which particles are scattering. Given a magic number \texttt{magic}, the three values can be computed in Listing~\ref{lst:magic}. Note that the group numbers are indexed from 1 to $G$ instead of 0 to $G-1$ as required by DOCTORS. This is corrected by subtracting 1 while doing the indexing arithmetic.

\begin{listing}
\begin{minted}[frame=lines,linenos]{c++}
magic = READ_NEXT_BINARY_INT()
KKK = magic % 1000
JJJ = (magic % 1000000 - KKK) /1000
III = (magic - JJJ - KKK) / 1000000

src = JJJ
while src >= III
	data = READ_NEXT_BINARY_FLOAT()
	xs[(src - 1)*G + KKK - 1] = data
	src = src - 1
\end{minted}
\caption{Computation of the source and sink groups from the magic number and the subsequent data parsing.}\label{lst:magic}
\end{listing}

Figure~\ref{fig:sigmacomp} shows the microscopic cross section data pulled from the 200-neutron/47-gamma group data file currently used by DOCTORS for hydrogen and oxygen. The data is compared to reference data pulled directly from the ENDF/B-VII.1 photoatomic (MT=501) data library~\citep{ref:cullend} accessible through the Sigma database~\citep{ref:sigma}.

\begin{figure}
    \centering
    \begin{subfigure}[b]{0.45\textwidth}
        \includegraphics[width=\textwidth]{figs/sigmacomp1}
        \caption{}
        \label{fig:ampx1}
    \end{subfigure}
    ~
    \begin{subfigure}[b]{0.45\textwidth}
        \includegraphics[width=\textwidth]{figs/sigmacomp2}
        \caption{}
        \label{fig:ampx2}
    \end{subfigure}
    \caption{The microscopic cross section in bars for hydrogen and oxygen. Both subfigures show identical data, (a) shows the entire data range available in the reference data library and (b) shows only the data range applicable to DOCTORS.}\label{fig:sigmacomp}
\end{figure}

\section{Generation of Material Cross Section Data}\label{sec:xsgen}

Once the data is parsed, weighted combinations of elemental data is used to create material cross sections. Cross section data for photons always uses the naturally ocurring since photo-atomic reactions are not sensitive to the nuclear differences between isotopes. The compositions of N materials are assumed to be given as weight fractions, $w_i$ subject to
\begin{equation}
\sum_{i = 1}^N w_i = 1
\end{equation}
and is converted to an atom fraction, $a_i$:
\begin{equation}
a_i = \frac{w_i}{M m_i}
\end{equation}
where
\begin{equation}
M = \sum_{i = 1}^N \frac{w_i}{m_i}
\end{equation}
and $m_i$ is the element's atomic weight.

In the solution employed by DOCTORS, only the total and scattering cross sections are required. The AMPX data files however, support arbitrary reaction types and have up to a dozen or more reactions for photo-atomic reactions alone for elemental datasets. The reaction types are identified by their MT designation. The total, inelastic, and elastic scatter cross sections are MT 501, 502, and 504 respectively. The scatter cross section used by DOCTORS is the sum of the elastic and inelastic cross sections.

The cross section of compound materials are computed as a weighted summation of their components. For example, the cross section of water for a particular reaction would be
\begin{equation}
\sigma_{H_2 O} = \frac{2 \sigma_H + \sigma_O}{3}.
\end{equation}

\begin{figure}
    \centering
    \begin{subfigure}[b]{0.45\textwidth}
        \includegraphics[width=\textwidth]{figs/airwaterxs2_19group}
        \caption{19 group data}
        \label{fig:airwaterxs2_19group}
    \end{subfigure}
    ~
    \begin{subfigure}[b]{0.45\textwidth}
        \includegraphics[width=\textwidth]{figs/airwaterxs2_47group}
        \caption{47 group data.}
        \label{fig:airwaterxs2_47group}
    \end{subfigure}
    \caption{Comparison of the group-averaged DOCTORS cross section data to continuous NIST data.}\label{fig:airwaterxs_group}
\end{figure}

\section{Qt5 Framework}\label{sec:qt}
The Qt5 framework was used for implementation of the user interface. Qt5 enables asynchronous calls through its signal/slot mechanism. Signals and slots are special functions that have additional processing performed by the meta-object-compiler (MOC). A signal can be emitted which will execute all connected slots. Signals and slots are connected manually by the user except special ones automatically generated by Qt5. 

Qt5 prvides a user interface for building user interfaces. Components such a sbuttons, drop boxes, radio buttons, etc. are made available to the user. Through extensive use of polymorphism, Qt simplifies the addition of graphical interactive elements called widgets. All widgets inherit from the base \texttt{QWidget} class which inherits from \texttt{QObject}. Any class that utilizes the signal/slot mechanism must extend the \texttt{QOjbect} class. As an example, a button can be created in the Qt5 user interface and named \texttt{button1}. This object will be accessible as \texttt{ui->button1}. When the user clicks on the button, the signal \texttt{button1.clicked()} will automatically be emitted. The user can connect this signal to any slot with an identical number arguments of the same type. The \texttt{button1.clicked()} signal can be connected to the slot \texttt{doStuff1()} but not \texttt{doStuff2(int)} since the arguments are not compatible.

Listing~\ref{lst:sync1} gives a pseudocode snippet that has a long function that will block the user interface. A corresponding dequence diagram is given in Fig.~\ref{fig:sync1}. When the user interacts with the GUI, the writer object begins executing the \texttt{doWrite()} function. Until this function completes, the UI thread will be busy an dunable to handle additionaly user interaction or updates. This results in the GUI becoming unresponsive and potentially issuing a warning to the user from the operating system.

\begin{listing}
\begin{minted}[frame=lines,linenos]{c++}
class MainWindow : public QMainWindow
{
	// Constructor
	MainWindow(QObject *parent);
	
	// Other parts of the class
	
protected slots:
	void doSomethingLong();
}

void MainWindow::doSomethingLong()
{
	// Execute a long piece of code
}
	

MainWindow::MainWindow(QObject *parent)
{
	// Initializations
	
	// When a button named button1 which was created in the Qt5
	creator interface is clicked, the clicked() signal is 
	automatically emitted which executes the doSomethingLong()
	function
	connect(ui->button1, clicked(), this, 
		doSomethingLong());
}
\end{minted}
\caption{A long function causes the user interface to block.}\label{lst:sync1}
\end{listing}

\begin{figure}[tb]
  \begin{center}
   \includegraphics[width=3.75in]{figs/writer_sync}
  \end{center}
  \caption{Sequence diagram for the synchronous call.}
\label{fig:sync1}
\end{figure}%

The code in Listing~\ref{lst:sync1} is updated to use the Qt signal/slot mechanism whose code is given in Listing~\ref{lst:async1}.

\begin{listing}
\begin{minted}[frame=lines,linenos]{c++}
class MainWindow : public QWindow
{
	// Member variables
	QThread workerThread;
	Worker *worker;
	
	// Constructor
	MainWindow(QObject *parent);
	
protected slots:
	handleResult();
}
	
MainWindow::MainWindow(QObject *parent)
{
	// Initializations
	worker = new Worker;
	
	worker.moveToThread(&workerThread);
	
	// Set up the thread connections
	connect(&workerThread, finished(), worker, deleteLater());
	connect(this, begin(), worker, doSomethingLong());
	connect(worker, done(), this, handleResult());
}

class Worker : public QObject
{
private signals:
	void done();
	
public slots:
	void soSomethingLong();
}

Worker::doSomethingLong()
{
	// Do stuff...
	emit done();
}

\end{minted}
\caption{Signals and slots enable a long function to be called without blocking the user interface.}\label{lst:async1}
\end{listing}

\begin{figure}[tb]
  \begin{center}
   \includegraphics[width=3.75in]{figs/writer_async}
  \end{center}
  \caption{Sequence diagram for the synchronous call.}
\label{fig:async1}
\end{figure}%

\section{Graphical User Interface}\label{sec:gui}

INSERT PICTURES OF THE GUI

\section{CUDA}\label{sec:cuda}
CUDA code is compoiled with the Nvidia complier nvcc. Qt5 uses the gcc compiler and its MOC generator for meta code. In order to connect CUDA code to the Qt MOC, the CUDA code is compiled by nvcc to produce a .o file. Qt5 then compiles all other files into corresponding.o files. The linker than automactically picks up all of the .o files gneerated by nvcc. The final result is an executable that has a Qt5 generarted user interface that can communicat with an Nvidia graphics card through the CUDA language.

The sweep through the mesh is different.

\begin{table}[ht]
\caption{Subsweep Indices}
\centering 
\begin{tabular}{l | c | c | c | c}
  \hline \hline   
  i  & ix & iy & iz & ix+iy+iz \\ [0.5ex] % inserts table 
  \hline
  0  & 4 & 0 & 0 & 4\\
  1  & 3 & 1 & 0 & 4\\
  2  & 3 & 0 & 1 & 4\\
  3  & 2 & 2 & 0 & 4\\
  4  & 2 & 1 & 1 & 4\\
  5  & 2 & 0 & 2 & 4\\
  6  & 1 & 3 & 0 & 4\\
  7  & 1 & 2 & 1 & 4\\
  8  & 1 & 1 & 2 & 4\\
  9  & 1 & 0 & 3 & 4\\
  10 & 0 & 4 & 0 & 4\\
  11 & 0 & 3 & 1 & 4\\
  12 & 0 & 2 & 2 & 4\\
  13 & 0 & 1 & 3 & 4\\
  14 & 0 & 0 & 4 & 4\\ [1ex]      % [1ex] adds vertical space
  \hline
\end{tabular}
\label{table:subsweep}
\end{table}

We first split the sweep process into $N_L$ subsweeps. For a $N_x \times N_x \times N_x$ mesh, the subsweep process is straightforward. Some of the initial subsweeps are shown for a $8 \times 8 \times 8$ mesh in Fig.~\ref{fig:subsweep_cube}. This is valid for any direction, $\hat{\Omega}$, for which all three of its direction cosines ($\mu$, $\xi$, and $\eta$) are positive.

\begin{figure}
    \centering
    \begin{subfigure}[b]{0.45\textwidth}
        \includegraphics[width=\textwidth]{figs/subsweep_cube1}
        \caption{}
        \label{fig:subsweep_cube1}
    \end{subfigure}
    ~ 
    \begin{subfigure}[b]{0.45\textwidth}
        \includegraphics[width=\textwidth]{figs/subsweep_cube2}
        \caption{}
        \label{fig:subsweep_cube2}
    \end{subfigure}
    ~ 
    \begin{subfigure}[b]{0.45\textwidth}
        \includegraphics[width=\textwidth]{figs/subsweep_cube3}
        \caption{}
        \label{fig:subsweep_cube3}
    \end{subfigure}
    \begin{subfigure}[b]{0.45\textwidth}
        \includegraphics[width=\textwidth]{figs/subsweep_cube4}
        \caption{) with labels}
        \label{fig:subsweep_cube4}
    \end{subfigure}
    \caption{The progression of subsweeps throughout a sweep. Each subsweep must complete before those after it. Each voxel within a subsweep can be solved in parallel with all others in its subsweep. (a) Subsweep 0 ($S=0$) (b) Subsweep 1 ($S=1$) (c) Subsweep 6 ($S=6$) (d)Subsweep 6 ($S=6$) with labels.}\label{fig:subsweep_cube}
\end{figure}

In general, blah blah.

\begin{figure}
    \centering
    \begin{subfigure}[b]{0.2\textwidth}
        \includegraphics[width=\textwidth]{figs/subsweep_general1}
        \caption{$S=0$}
        \label{fig:subsweep_general1}
    \end{subfigure}
    ~ %add desired spacing between images, e. g. ~, \quad, \qquad, \hfill etc. 
      %(or a blank line to force the subfigure onto a new line)
    \begin{subfigure}[b]{0.2\textwidth}
        \includegraphics[width=\textwidth]{figs/subsweep_general2}
        \caption{$S=1$}
        \label{fig:subsweep_general2}
    \end{subfigure}
    ~ %add desired spacing between images, e. g. ~, \quad, \qquad, \hfill etc. 
    %(or a blank line to force the subfigure onto a new line)
    \begin{subfigure}[b]{0.2\textwidth}
        \includegraphics[width=\textwidth]{figs/subsweep_general3}
        \caption{$S=2$}
        \label{fig:subsweep_general3}
    \end{subfigure}
    \begin{subfigure}[b]{0.2\textwidth}
        \includegraphics[width=\textwidth]{figs/subsweep_general4}
        \caption{$S=3$}
        \label{fig:subsweep_general4}
    \end{subfigure}
    
    \begin{subfigure}[b]{0.2\textwidth}
        \includegraphics[width=\textwidth]{figs/subsweep_general5}
        \caption{$S=4$}
        \label{fig:subsweep_general5}
    \end{subfigure}
    ~ %add desired spacing between images, e. g. ~, \quad, \qquad, \hfill etc. 
      %(or a blank line to force the subfigure onto a new line)
    \begin{subfigure}[b]{0.2\textwidth}
        \includegraphics[width=\textwidth]{figs/subsweep_general6}
        \caption{$S=5$}
        \label{fig:subsweep_general6}
    \end{subfigure}
    ~ %add desired spacing between images, e. g. ~, \quad, \qquad, \hfill etc. 
    %(or a blank line to force the subfigure onto a new line)
    \begin{subfigure}[b]{0.2\textwidth}
        \includegraphics[width=\textwidth]{figs/subsweep_general7}
        \caption{$S=6$}
        \label{fig:subsweep_general7}
    \end{subfigure}
    \begin{subfigure}[b]{0.2\textwidth}
        \includegraphics[width=\textwidth]{figs/subsweep_general8}
        \caption{$S=7$}
        \label{fig:subsweep_general8}
    \end{subfigure}
    
    \begin{subfigure}[b]{0.2\textwidth}
        \includegraphics[width=\textwidth]{figs/subsweep_general9}
        \caption{$S=8$}
        \label{fig:subsweep_general9}
    \end{subfigure}
    ~ %add desired spacing between images, e. g. ~, \quad, \qquad, \hfill etc. 
      %(or a blank line to force the subfigure onto a new line)
    \begin{subfigure}[b]{0.2\textwidth}
        \includegraphics[width=\textwidth]{figs/subsweep_general10}
        \caption{$S=9$}
        \label{fig:subsweep_general10}
    \end{subfigure}
    ~ %add desired spacing between images, e. g. ~, \quad, \qquad, \hfill etc. 
    %(or a blank line to force the subfigure onto a new line)
    \begin{subfigure}[b]{0.2\textwidth}
        \includegraphics[width=\textwidth]{figs/subsweep_general11}
        \caption{$S=10$}
        \label{fig:subsweep_general11}
    \end{subfigure}
    \begin{subfigure}[b]{0.2\textwidth}
        \includegraphics[width=\textwidth]{figs/subsweep_general12}
        \caption{$S=11$}
        \label{fig:subsweep_general12}
    \end{subfigure}
    
    \begin{subfigure}[b]{0.2\textwidth}
        \includegraphics[width=\textwidth]{figs/subsweep_general13}
        \caption{$S=12$}
        \label{fig:subsweep_general13}
    \end{subfigure}
    ~ %add desired spacing between images, e. g. ~, \quad, \qquad, \hfill etc. 
      %(or a blank line to force the subfigure onto a new line)
    \begin{subfigure}[b]{0.2\textwidth}
        \includegraphics[width=\textwidth]{figs/subsweep_general14}
        \caption{$S=13$}
        \label{fig:subsweep_general14}
    \end{subfigure}
    ~ %add desired spacing between images, e. g. ~, \quad, \qquad, \hfill etc. 
    %(or a blank line to force the subfigure onto a new line)
    \begin{subfigure}[b]{0.2\textwidth}
        \includegraphics[width=\textwidth]{figs/subsweep_general15}
        \caption{$S=14$}
        \label{fig:subsweep_general15}
    \end{subfigure}
    \caption{The generalized subsweep.}\label{fig:subsweep_general}
\end{figure}

The number of parallel tasks, $P$, that can be done on subsweep $S$ of an $N_x \times N_y \times N_z$ mesh is given by Eq.~\ref{eq:taskspersub}.

\begin{equation}\label{eq:taskspersub}
P = C_S - L_x - L_y - L_z + G_{xy} + G_{yz} + G_{xz}
\end{equation}

\begin{equation}
C_S = \frac{(S+1)(S+2)}{2}
\end{equation}

\begin{equation}
L_x = \frac{d_x(d_x+1)}{2}
\end{equation}

\begin{equation}
L_y = \frac{d_y(d_y+1)}{2}
\end{equation}

\begin{equation}
L_z = \frac{d_z(d_z+1)}{2}
\end{equation}

\begin{equation}
G_{xy} = \frac{d_{xy}(d_{xy}+1)}{2}
\end{equation}

\begin{equation}
G_{yz} = \frac{d_{yz}(d_{yz}+1)}{2}
\end{equation}

\begin{equation}
G_{xz} = \frac{d_{xz}(d_{xz}+1)}{2}
\end{equation}

\begin{equation}
d_x = \max(S+1-N_x, 0)
\end{equation}

\begin{equation}
d_y = \max(S+1-N_y, 0)
\end{equation}

\begin{equation}
d_z = \max(S+1-N_z, 0)
\end{equation}

\begin{equation}
d_{xy} = \max(S+1-N_x - Ny, 0)
\end{equation}

\begin{equation}
d_{yz} = \max(S+1-N_y - Nz, 0)
\end{equation}

\begin{equation}
d_{xz} = \max(S+1-N_x - Nz, 0)
\end{equation}

Each voxel in a subsweep can be computed in parallel. Mathematically, the $i^{th}$ subsweep from all directions can be computed in parallel. However, in practice, this results in a race condition on the GPU hardware.


DELETE BELOW HERE.

The number of elements before the 4$^{th}$ level is 1 + 2 + 3 = 6. This is simply the sum of all integers less than the level number given by Eq.~\ref{eq:levelsum}. This is conincidentally the index of the first element of that level. However, there is a special case of $N_0 = 1$.

\begin{equation} \label{eq:levelsum}
N_L = \sum_{i=1}^{L} = \frac{L(L+1)}{2}
\end{equation}

Given an index, $i$, the level to which it belongs can be computed by substituting $i$ for $N_L$ in Eq.~\ref{eq:levelsum} and solving for $L$. The index value is then floored.

\begin{equation} \label{eq:levelquadratic}
i = \frac{L(L+1)}{2} \rightarrow L^2 + L - 2i = 0
\end{equation}

\begin{equation} \label{eq:levelquadsol1}
L = \left \lfloor{\frac{-1 \pm \sqrt{1+8i}}{2}} \right \rfloor
\end{equation}

Since the index must be a positive value, the $\pm$ sign can be removed from Eq.~\ref{eq:levelquadsol1} yielding Eq.~\ref{eq:levelquadsol2}

\begin{equation} \label{eq:levelquadsol2}
L = \left \lfloor{\frac{-1 + \sqrt{1+8i}}{2}} \right \rfloor
\end{equation}

The maximum index of any level in a the $N$ subsweep is $N$. Each subsweep adds an additional level. Therefore, the ix, iy, and iz indices can be computed using Eq.~\ref{eq:levelix},~\ref{eq:leveliy}, and~\ref{eq:leveliz} respectively.

\begin{equation} \label{eq:levelix}
ix = N - N_L
\end{equation}

\begin{equation} \label{eq:leveliy}
iy = L + N_L - i
\end{equation}

\begin{equation} \label{eq:leveliz}
iz = L + i - N_L
\end{equation}

\section{MCNP Generation}\label{sec:mcnpgen}
DOCTORS has the capability to generate MCNP6 input files from the CT mesh data and source specification provided by the user.

\section{Hardware}\label{sec:hardware}
For this work, a computer with an Intel i7-5960X 8 core (16 hyperthreads) processor with a base clock speed of 3.X GHz and an Nvidia Titan Z graphics card was used. Currently, if the problem requires more memory than is available on the GPU, the problem can still be solved, but much more time will be required to to copy overhead between the CPU and GPU. If the problem requires more memory than either the GPU or CPU can provide, an error is thrown and the simulation is not run.

CUDA is Nvidia only so AMD is not considered.

\endinput
%%
%% End of file `chapall.tex'.
